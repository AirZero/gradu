\documentclass[utf8,bachelor]{gradu3}

\usepackage{graphicx} % kuvien mukaan ottamista varten
\usepackage{booktabs} % hyvä kauniiden taulukoiden tekemiseen
% HUOM! Tämän tulee olla viimeinen \usepackage koko dokumentissa!
\usepackage[bookmarksopen,bookmarksnumbered,linktocpage]{hyperref}

\addbibresource{eetu_rantakangas_kandi.bib} % Lähdetietokannan tiedostonimi
\begin{document}
\title{Innovatiiviset opetusmenetelmät}
\translatedtitle{Innovative teaching methods}
\studyline{Kaikki suuntautumisvaihtoehdot}
\avainsanat{Opetusmenetelmä, innovaatio}
\keywords{Education method, innovation}
\tiivistelma{
  %Tutkielman tiivistelmä on tyypillisesti lyhyt esitys, jossa kerrotaan tutkielman taustoista, tavoitteesta, tutkimusmenetelmistä, saavutetuista tuloksista, tulosten tulkinnasta ja johtopäätöksistä. Tiivistelmän tulee olla niin lyhyt, että se, englanninkielinen abstrakti ja muut metatiedot mahtuvat kaikki samalle sivulle.

}
\abstract{
%TODO tähän englantia
}

\author{Eetu Rantakangas}
\contactinformation{\texttt{eetu.rantakangas@iki.fi}}
% jos useita tekijöitä, anna useampi \author-komento
\supervisor{Sanna Mönkölä}
% jos useita ohjaajia, anna useampi \supervisor-komento

\maketitle

\mainmatter

\chapter{Johdanto}

%TODO Johdanto

\chapter{Tutkimussuunnitelma}



%TODO Hanki rdomilta paketti joka handlaa urlit
\printbibliography

\end{document}
